\documentclass[usenatbib]{mnras}
\pdfoutput=1
\pdfminorversion=5

%\usepackage{amsmath}
\usepackage{mathtext,amssymb,amsmath}
\usepackage{epsfig}
\usepackage{graphics}
\usepackage{url}
% If your system does not have the AMS fonts version 2.0 installed, then
% remove the useAMS option.
%
% useAMS allows you to obtain upright Greek characters.
% e.g. \umu, \upi etc.  See the section on "Upright Greek characters" in
% this guide for further information.
%
% If you are using AMS 2.0 fonts, bold math letters/symbols are available
% at a larger range of sizes for NFSS release 1 and 2 (using \boldmath or
% preferably \bmath).
%
% The usenatbib command allows the use of Patrick Daly's natbib.sty for
% cross-referencing.
%
% If you wish to typeset the paper in Times font (if you do not have the
% PostScript Type 1 Computer Modern fonts you will need to do this to get
% smoother fonts in a PDF file) then uncomment the next line
\usepackage{times}
%\usepackage{helvet}
\usepackage[T1]{fontenc} 
\usepackage{aecompl}

%%%%% AUTHORS - PLACE YOUR OWN MACROS HERE %%%%%

\renewcommand{\vector}[1]{\ensuremath{\mathbf{#1}}}
\renewcommand{\div}{\ensuremath{-}}

\newcommand{\bea}{\begin{eqnarray}}
\newcommand{\eea}{\end{eqnarray}}
  
\newcommand{\Mach}{\ensuremath{\mathcal{M}}}
\newcommand{\rot}{\ensuremath{\mathbf{curl\,}}}
\newcommand{\mdot}{\ensuremath{\dot{m}}}
\newcommand{\Msun}{\ensuremath{\,\rm M_\odot}}
\newcommand{\Msunyr}{\ensuremath{\,\rm M_\odot\, \rm yr^{-1}}}
\newcommand{\ergl}{\ensuremath{\,\rm erg\, s^{-1}}}
\newcommand{\Gyr}{\ensuremath{\,\rm Gyr}}
\newcommand{\yr}{\ensuremath{\,\rm yr}}
\newcommand{\mum}{\ensuremath{\,\rm \mu m}}
\newcommand{\pc}{\ensuremath{\,\rm pc}}
\newcommand{\cmc}{\ensuremath{\,\rm cm^{-3}}}
\newcommand{\cmsq}{\ensuremath{\,\rm cm^{-2}}}

\newcommand{\Ry}{\ensuremath{\,{\rm Ry}}}
\newcommand{\eV}{\ensuremath{\,{\rm eV}}}
\newcommand{\keV}{\ensuremath{\,{\rm keV}}}
\newcommand{\GeV}{\ensuremath{\,{\rm GeV}}}
\newcommand{\TeV}{\ensuremath{\,{\rm TeV}}}
\newcommand{\AAA}{\ensuremath{\,{\rm \AA}}}
% \newcommand{\ion}[2]{ \ensuremath{ \rm #1\,\sc #2 }}

\newcommand{\acos}{\ensuremath{\rm acos}\,}
\newcommand{\litwo}{\ensuremath{\rm Li}_2\,}
\newcommand{\lithree}{\ensuremath{\rm Li}_3\,}
\newcommand{\li}[2]{{\rm Li}_{#1}\!\left(#2\right)}
\newcommand{\gf}{\ensuremath{\frac{\sqrt{g_{\varphi\varphi}}}{\alpha}}}
\newcommand{\pardir}[2]{\ensuremath{\frac{\partial #2}{\partial #1} }}
\newcommand{\ppardir}[2]{\ensuremath{\frac{\partial }{\partial #1} \left( #2\right)}}
\newcommand{\eps}{\epsilon}

%\newcommand{\tcfour}{3C\,454.3}
%\newcommand{\tctwo}{3C\,279}
%\newcommand{\pkst}{PKS 1222$+$21}
%\newcommand{\pksf}{PKS 1510$-$89}
% \newcommand{\grs}{GRS~1915+105}

\newcommand{\cloudy}{{\sc cloudy}}
\newcommand{\xstar}{{\sc xstar}}

%%%%%%%%%%%%%%%%%%%%%%%%%%%%%%%%%%%%%%%%%%%%%%%%

\title[]{Time-dependent magnetospheric flow}
\author[]{}

\begin{document}

\date{Accepted ---. Received ---; in
  original form --- }

\label{firstpage}
\pagerange{\pageref{firstpage}--\pageref{lastpage}} \pubyear{2016}
\maketitle

\begin{abstract}

\end{abstract}

\begin{keywords}
quasars: emission lines -- gamma-rays: general -- opacity
\end{keywords}

\section{Dimensionless formulation}

Let us assume $GM_*=c=\varkappa=1$, where $M_*$ is the mass of the accretor,
and $\varkappa$ is the opacity
(cm$^2$\,g$^{-1}$, assumed constant). We have the right. This implies the
following units for time, length, mass, and energy:
\begin{equation}\label{E:timedep:time}
  \Delta t = \frac{GM}{c^3},
\end{equation}
\begin{equation}\label{E:timedep:length}
  \Delta l = \frac{GM}{c^2},
\end{equation}
\begin{equation}\label{E:timedep:mass}
  \Delta m = \frac{G^2M^2}{\varkappa c^4},
\end{equation}
and
\begin{equation}\label{E:timedep:energy}
  \Delta E = \frac{G^2M^2}{\varkappa c^2}.
\end{equation}
Luminosity is then scaled to Eddington units, more precisely 
\begin{equation}\label{E:timedep:luminosity}
  \Delta L = \frac{GM c}{\varkappa} =  \frac{L_{\rm Edd}}{4\pi}.
\end{equation}
For a one-dimensional formulation, let us integrate in the direction
perpendicular to the magnetic field lines. If the strip we are integrating
over is narrow, the cross-section of the area we
integrate over is
\begin{equation}\label{E:timedep:across}
\displaystyle  A = 4\pi a R_{\rm e} dR_{\rm e} \frac{\sin^3\theta
}{\sqrt{1+3\cos^2\theta}}.
\end{equation}
Here, $0<a\leq 1$ is the part of the full $2\pi$ azimuthal extent subtented by
the flow. As there is no perfect axisymmetry, we expect the magnetospheric
flow exist only at certain longitudes. 
Computation involves a conservative scheme for the three conserved quantities,
mass, momentum along the field line, and energy, expressed per unit length
along the field line:
\begin{equation}\label{E:timedep:m}
  m = \int \rho dS = \pardir{l}{M},
\end{equation}
\begin{equation}\label{E:timedep:s}
  s = \int \rho v dS = \pardir{l}{p},
\end{equation}
\begin{equation}\label{E:timedep:e}
\displaystyle  e = \int \left(u + \rho \left( \frac{v^2}{2} -\frac{1}{r} - \frac{1}{2} \omega^2 r^2\sin^2\theta \right) \right) dS = \pardir{l}{E}.
\end{equation}
For each of the three quantities, conservation laws have the general form
\begin{equation}\label{E:timedep:qcons}
  \pardir{t}{q} + \pardir{l}{F_q} = S_q,
\end{equation}
where $F_q$ and $S_q$ are, respectively, the flux and source for the particula
quantity. Fluxes
\begin{equation}\label{E:timedep:fluxm}
F_m =  \int \rho v dS = s,
\end{equation}
\begin{equation}\label{E:timedep:fluxs}
F_s =  \int \left(\rho v^2 + p\right) dS,
\end{equation}
and
\begin{equation}\label{E:timedep:fluxe}
F_e =  \int \left( v \rho \left(\frac{u + p}{\rho} + \frac{v^2}{2} -\frac{1}{r} -
\frac{1}{2} \omega^2 r^2\sin^2\theta \right) - D \frac{du}{dl}\right) dS,
\end{equation}
where the pressure $p=u/3$, as we consider only radiation-dominated case, and
$D = \frac{1}{3\rho}$ is diffusion coefficient. We do not consider any sources
or losses of mass ($S_m=0$), while for momentum, gravitational and centrifugal forces
were taken into account
\begin{equation}\label{E:timedep:srcs}
S_s = - \frac{1}{r^2}\sin(\theta+\alpha) + \omega^2r \sin\theta \cos\alpha.
\end{equation}
For energy, there are two contributions: work done by the forces and energy
loss due to radiation,
\begin{equation}\label{E:timedep:srce}
  S_e = v S_s - \xi_{\rm rad} a r \sin\theta \frac{u}{\rho+1}.
\end{equation}
\alert{maybe a good idea is to introduce a bulk viscousity term?\\}

Resulting system of three differential equations was then solved using HLLE
Riemann solver (see for instance \citealt{einfeldt}) with the signal
velocities fixed by $v\pm 1$. No relativistic effects were taken into
account. 

Boundary conditions were profoundly different from those in \citet{basko-sunyaev1976}: we
assume no matter or energy flow through the lower boundary (NS surface), and a constant
mass inflow at a fixed velocity at the right (disc) boundary. 




\label{lastpage}

\end{document}
